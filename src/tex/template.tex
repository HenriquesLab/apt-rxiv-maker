% Rxiv-Maker Article Template
% A modern LaTeX template for scientific articles

\documentclass[11pt,a4paper]{article}

% Import the rxiv-maker style package
\usepackage{rxiv-maker}

% Additional packages for scientific writing
\usepackage[utf8]{inputenc}
\usepackage[T1]{fontenc}
\usepackage{lmodern}
\usepackage{microtype}
\usepackage{natbib}
\usepackage{url}
\usepackage{doi}

% Document metadata
\title
\author
\date

% Abstract and keywords (if defined)
\newcommand{\abstract}[1]{\begin{abstract}#1\end{abstract}}
\newcommand{\keywords}[1]{\textbf{Keywords:} #1}

\begin{document}

\maketitle

% Abstract section (replace with actual content)
\begin{abstract}
This is a template abstract. Replace this text with your actual abstract content.
The abstract should provide a concise summary of your research, methodology, 
and key findings.
\end{abstract}

\keywords{template, LaTeX, scientific writing, rxiv-maker}

\section{Introduction}

This is a template document created by rxiv-maker. Replace this content with 
your actual article text. This template provides a clean, modern layout 
suitable for scientific publications.

\section{Methods}

Describe your methodology here. You can include equations like:

\begin{equation}
E = mc^2
\label{eq:einstein}
\end{equation}

Reference the equation as \cref{eq:einstein}.

\section{Results}

Present your results here. You can include figures and tables:

\begin{figure}[htbp]
    \centering
    % \includegraphics[width=0.8\textwidth]{figure1.pdf}
    \caption{This is a placeholder for a figure. Replace with your actual figure.}
    \label{fig:example}
\end{figure}

\section{Discussion}

Discuss your findings and their implications here.

\section{Conclusion}

Summarize your work and key conclusions.

\section*{Acknowledgments}

Acknowledge funding sources, collaborators, and other contributions.

% Bibliography
\bibliographystyle{plain}
% \bibliography{references}

\end{document}